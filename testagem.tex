\chapter{Testagem dos resultados obtidos}
\section{Metodologia de testagem}
Ao longo da criação deste script é necessária uma constante
testagem das funções, para garantir que as funções estão
corretamente implementadas, que alterações a partes
distintas do código não afetam as já implementadas e para
sabermos o progresso já efetuado.

De modo a facilitar o processo de desenvolvimento, podemos
sempre automatizar esta testagem. Para tal, temos que criar diretórios e ficheiros com nomes,
tamanhos e datas de modificação distintos. Depois podemos
executar o nosso script com diversos argumentos e comparar o
seu output com um ficheiro pre-feito.

Foi-nos também fornecido pelos docentes da unidade
curricular um ficheiro de teste, permitindo-nos testar a
mais importante função do nosso script, a contabilização do
espaço. Tendo a confirmação de que o espaço ocupado é
corretamente contabilizado por ficheiro e diretoria, podemos
focar a nossa atenção para todas as outras funcionalidades,
desde seleção dos ficheiros por nome, data de modificação ou
tamanho ao modo de ordenamento dos resultados.

\section{Comandos e programas úteis para testagem}
Tendo então a necessecidade de criação de diretórios,
necessitaremos de usar o comando \verb|mkdir|. Para a
criação de ficheiros com tamanho não nulo,
utilizaremos \verb|truncate|, modificando depois a data de
última modificação com \verb|touch|. Por fim, para comparar
o resultado obtido com um resultado expectado, utilizaremos
o comando \verb|diff|, que dependendo do valor retornado nos
permite saber se os ficheiros são ou não iguais.

\section{Funções auxiliares}

Para facilitar o processo de criação de testes, procedemos à criação de duas funções auxiliares: \verb|create_file| e
\verb|test_passed|.

Assumamos que \verb|create_file| aceita por argumento o tamanho do ficheiro a criar, a data da última modificação e o nome do
ficheiro a criar.
\begin{minted}{bash}
create_file() {
    truncate -s "$1" "./unit_tests/$3"
    #   Criação de um ficheiro com tamanho predefinido
    touch -d "$2" "./unit_tests/$3"
    #   Alteração da data de última modificação
}
\end{minted}

Por fim, criamos diversos ficheiros e diretórios
\begin{listing}[H]
\begin{minted}{bash}
    create_file "14M" "Nov 1" "file1.a"
    create_file "2" "Sep 10" "file\ with\ spaces.spacedout"
    create_file "4M" "Dec 21" "file\ttabbed"
    create_file "3M" "Nov 1" "under_scored"
    create_file "1M" "Nov 1" "hyphened-file"
    create_file "523M" "Nov 1" "a-script.c"
    create_file "11M" "Nov 1" "having_extensions.sh"
    
    mkdir ./unit_tests/directoryA
    create_file "1M" "Nov 1" "directoryA/file1.a"
    create_file "2K" "Sep 10" "directoryA/file\ with\ spaces.spacedout"
    create_file "3M" "Dec 21" "directoryA/file\ttabbed"
    create_file "16M" "Nov 1" "directoryA/under_scored"
    create_file "1K" "Nov 1" "directoryA/hyphened-file"
    create_file "53M" "Nov 1" "directoryA/a-script.c"
    create_file "21M" "Nov 1" "directoryA/having_extensions.sh"
    
    mkdir ./unit_tests/aaaaaaa
    create_file "2M" "Sep 10" "aaaaaaa/file\ with\ spaces.spacedout"
    create_file "30K" "Nov 1" "aaaaaaa/under_scored"
    create_file "21K" "Nov 1" "aaaaaaa/hyphened-file"
    create_file "523K" "Nov 1" "aaaaaaa/a-script.c"
    create_file "210K" "Nov 1" "aaaaaaa/having_extensions.sh"
\end{minted}
\caption{Exemplo da criação de ficheiros e diretórios}
\end{listing}
E podemos agora chamar diversas vezes o script \textbf{spacecheck.sh} com quaisquer combinações de argumentos.
