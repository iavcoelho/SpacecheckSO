\chapter{Testagem dos resultados obtidos}
\section{Metodologia de testagem}
Ao longo da criação deste script é necessária uma constante
testagem das funções, para garantir que as funções estão
corretamente implementadas, que alterações a partes
distintas do código não afetam as já implementadas e para
sabermos o progresso já efetuado. Como tal, consoante vamos
desenvolvendo o nosso script, vamos também desenvolvendo mais e
mais testes, obtendo no final uma suíte de testes compreensiva
que visa testar todas as funcionalidades do programa de forma
extensiva, tentando apanhar quaisquer erros e exceções que o
script possa encontrar no mundo real.

De modo a facilitar o processo, podemos
sempre automatizar esta testagem, permitindo assim reduzir
substancialmente o tempo gasto em testagem.
Para tal, temos que criar diretórios e ficheiros com nomes,
tamanhos e datas de modificação distintos. Depois podemos
executar o nosso script com diversos argumentos e comparar o
seu output com um ficheiro pré-feito.

Foi-nos também fornecido pelo docente da unidade
curricular um ficheiro de teste, permitindo-nos comparar a 
contabilização do espaço ocupado com um valor que sabemos ser 
correto. Tendo a confirmação de que o espaço ocupado é
corretamente contabilizado por ficheiro e diretoria, podemos
focar a nossa atenção para todas as outras funcionalidades,
desde seleção dos ficheiros por nome, data de modificação ou
tamanho ao modo de ordenamento dos resultados.

É, no entanto, importante notar que \textbf{o nosso script não é 100\%
determinístico}, devido à presença da data no cabeçalho do
output. Uma vez que alguns erros impedem a impressão do
cabeçalho, não podemos apenas remover a primeira linha do
output, pelo que utilizaremos a biblioteca \verb|faketime| para
alterar a data utilizada pelo script sem necessidade de alterar
a data do computador.

\section{Comandos e programas úteis para testagem}
Tendo então a necessidade de criação de diretórios,
necessitaremos de usar o comando \verb|mkdir|. Para a
criação de ficheiros com tamanho não nulo,
utilizaremos \verb|truncate|, modificando depois a data de
última modificação com \verb|touch|. Devido à possibilidade de
ser possível encontrar diversos ficheiros e diretórios aos
quais o utilizador não têm permissões de leitura ou acesso,
então utilizaremos também o \verb|chmod| para simular essas
situações. Por fim, para comparar o resultado obtido com um 
resultado expectado, utilizaremos o comando \verb|diff|, 
que dependendo do valor retornado permite-nos saber se os
ficheiros são ou não iguais.

\section{Funções auxiliares}

Para facilitar o processo de criação de testes, procedemos à criação de duas funções auxiliares: \verb|create_file| e
\verb|test_passed|. No final, queremos também saber se todos os testes foram bem-sucedidos ou não, tendo então a função
\verb|final_result|.

Comecemos pela criação de ficheiros. Assumamos que \verb|create_file| aceita por argumento o tamanho do ficheiro a
criar, a data da última modificação e o nome do ficheiro a criar.
\begin{minted}{bash}
create_file() {
    truncate -s "$1" "./unit_tests/$3"
    #   Criação de um ficheiro com tamanho predefinido
    touch -d "$2" "./unit_tests/$3"
    #   Alteração da data de última modificação
}
\end{minted}

Depois, para verificar se um dado teste foi bem-sucedido, podemos só analisar o código de retorno do comando. Depois
podemos imprimir uma mensagem e atualizar uma variável \verb|all_tests_passed|.

\begin{minted}{bash}
test_passed() {
    passed_all_tests=$(($passed_all_tests + $?))
    if [[ $? == 0 ]]; then
        echo "Test $1 passed!"
    else
        echo "Test $1 failed!"
    fi
}
\end{minted}

Depois, precisamos de no final do programa verificar se todos os testes foram bem-sucedidos ou não, tendo então a função
\verb|final_result| para facilitar o processo.

\begin{minted}{bash}
final_result() {
    if [[ $passed_all_tests == "0" ]]; then
        echo "All tests passed!"
    else
        echo "Tests failed!"
    fi
}
\end{minted}

Por fim, criamos diversos ficheiros e diretórios
\begin{listing}[H]
\begin{minted}{bash}
    create_file "14M" "Nov 1" "file1.a"
    create_file "2" "Sep 10" "file\ with\ spaces.spacedout"
    create_file "523M" "Nov 1" "a-script.c"
\end{minted}
\caption{Exemplo da criação de ficheiros e diretórios}
\end{listing}
E podemos agora chamar diversas vezes o script \textbf{spacecheck.sh} com quaisquer combinações de argumentos, incluindo
algumas que propositadamente erradas, de modo a testar o error handling do nosso script.
